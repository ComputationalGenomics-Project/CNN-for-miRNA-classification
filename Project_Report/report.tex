% Project report for Computational Genomics s2017
% By Han Den (Andrew Id: hdeng1) and Luyi Ma (Andrew Id: luyim)
% Topic : Classification of miRNA using deep-learning method

\documentclass[letterpaper, 11pt]{article}
\usepackage[left=1in, right=1in, top=1in, bottom=1in]{geometry}
\usepackage{multicol}
\setlength{\columnsep}{1cm}


\begin{document}

% title
\begin{center}
{\Large
	\textsc{\textbf Classification of microRNA Using Deep-learning Method}
}

\vspace{0.3cm}

\normalsize Project report of Computational Genomics Spring 2017

\vspace{0.5cm}

{\small
	Han Den (Andrew Id: hdeng1)
	
	Luyi Ma (Andrew Id: luyim)
}
\end{center}

% abstract
\begin{abstract}
type some things here

\vspace{2mm}
\bfseries{ Keywords: miRNA, Convolution Neural Network, Classification}
\end{abstract}

\begin{multicols*}{2}
% introduction
\section{Introduction}
{
State the motivation, the problem you are addressing, and your approach for solving the problem. Use citations to provide a overview of the recent literature. It may be helpful to read a relevant review article.
\newline 
// TODO: fill this block

\# some possible topics

$\bullet$ miRNA structure

$\bullet$ miRNA database (introduction of this paper \textbf{miRFam: an effective automatic miRNA classification method based on n-grams and a multiclass SVM)}

$\bullet$ sequence recognizion and classification , some methods? CNN in sequence analysis

$\bullet$ Motivation of this project, next step is to classify miRNA sequence into subclasses...

}
\section{Methods}
{
Explain your computational approach. Describe your model and learning/inference methods in two different subsections. Define all variables and include self-sufficient equations.
\newline 
// TODO: fill this block

$\bullet$ sequence vectorization

$\bullet$ CNN structure for this classification job.

}

\section{Implementation details}
{
Give enough detail so the results can be reproduced by someone familiar with the field. Include a description of data processing steps and how you selected constants and/or free parameters (if applicable). Include pseudocode if implementing a new algorithm.
\newline
// TODO: fill this block
}

\section{Results and Conclusions}
{
Provide informative figures and legends, a summary of conclusions, limitations and future directions.
\newline 
// TODO: fill this block
}
\end{multicols*}

\newpage
\begin{multicols*}{2}

\begin{thebibliography}{1}
\bibitem{reference}
hahahaha, haliluya
\end{thebibliography}

\end{multicols*}

\end{document}
