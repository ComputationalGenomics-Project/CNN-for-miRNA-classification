% Project proposal for Computational Genomics s2017
% By Han Den (Andrew Id: hdeng1) and Luyi Ma (Andrew Id: luyim)
% Topic : Improving performance of Random Forest in Clinical Feature Learning

\documentclass[letterpaper, 12pt]{article}

%package
\usepackage[top= 0.5in, left = 1in, right= 1in, bottom= 0.5in]{geometry}
\usepackage{titlesec}
\usepackage{color}

\titleformat*{\section}{\large\bfseries\rmfamily}
\titleformat*{\subsection}{\large\bfseries}
\titleformat*{\subsubsection}{\large\bfseries}
\titleformat*{\paragraph}{\large\bfseries}
\titleformat*{\subparagraph}{\large\bfseries}
\renewcommand{\labelitemi}{}
\renewcommand{\labelitemii}{}


\begin{document}

\begin{center}
{\Large
	\textsc{\textbf Medical Images Feature Learning and Analysis with Deep Learning and Random Forest}
}

\vspace{0.3cm}

\normalsize Project proposal of Computational Genomics Spring 2017

\vspace{0.5cm}

{\small
	Han Den (Andrew Id: hdeng1)
	
	Luyi Ma (Andrew Id: luyim)
}
\end{center}

%body of the proposal
\section{Project Idea}

%Project idea description
\begin{itemize}
\item
Tuberculosis disease diagnosis can be aprroached by analyzing clinical symptoms and medical signs. With the help of CT and X-ray images, the process of tuberculosis diagnosis can be quantified. However, before analyzing these images, features for diagnosing tuberculosis must be extracted. Traditionally, it is done manually. In our project, we decide to apply deep learning and random forest methods to approach feature extraction of CT and X-ray images and "diagnose" diseases based on these features. We will use part of our dataset to train our model. 
\color{red} Ideally, the deep learning network will be worked as a feature extraction layer and the outcoming features will be ... \$ not determine yet ..\$)
\color{black} The performance of feature extraction will be evaluated by using these features to classify a test dataset. 

CT and X-ray images of tuberculosis patients from Belarus tuberculosis portal
~\cite{TB_database}.
It is an open-source database and both CT scan images and chest X-ray images can be found by searching the patient's ID. Clinical records of each patient are also provided.
\end{itemize}

%Software design or algorithm to develop
\section{Software Design}
\begin{itemize}
\item
We plan to use Python as our prefered programming language. We will implement algorithms involved in deep learning and random forest in Python.

\color{red} (deep learning framework ? \$ Tenserflow ? opencv ? \$)

\end{itemize}


%Papers to read
\section{Papers to Read}
\begin{itemize}
\item
TYPE papers to read here.
\end{itemize}


%reference
\begin{thebibliography}{1}
\bibitem{TB_database}
Open access information resources and the possibility of image-based detection of multiresistant tuberculosis, http://obsolete.tuberculosis.by
\end{thebibliography}
\end{document}